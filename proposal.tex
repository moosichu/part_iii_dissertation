\documentclass[11pt]{article}
\usepackage{a4wide,parskip,times}
\usepackage{cite}
\usepackage{url}

\begin{document}

\centerline{\Large Heterogeneous type checking in CPU-GPU systems}
\vspace{2em}
\centerline{\Large \emph{A PartIII project proposal}}
\vspace{2em}
\centerline{\large T. M. Read Cutting (\emph{tr395}), Downing College}
\vspace{1em}
\centerline{\large Project Supervisor: Prof A. Mycroft}
\vspace{1em}

\begin{abstract}

Programming for the GPU can be complex, error-prone and hard to manage using
existing tool chains, as programmers are required to write code in different
languages for the GPU and CPU, with little to no runtime or compile-time safety
across the boundaries between them. Various research has been done into
creating and designing unified programming languages which can be compiled to
heterogeneous architectures with systems to allocate load at runtime and handle
parallelisation and vectorisation automatically, greatly simplifying the
programming workflow. However, these languages are either domain specific or
have otherwise yet to experience serious uptake in soft real-time computing due
to various factors, including runtime overhead, lack of low-level control, and
other inefficiencies which make them unsuitable for soft real-time
applications. The result of this is that existing tool chains have not
developed alongside the existing research which aims to target systems of the
future as opposed to acknowledging the current state of affairs.

This project proposes a pragmatic solution of using a hybrid system composed of
two programming languages which each compile down to their respective
architectures (CPU/GPU), similar to existing workflows. This project
contributes to the field in providing a unified type-checking system across the
two languages in order to eliminate common errors - with the ability to use
type-safe syntactic sugar that compiles down to the boilerplate code that
handles API calls across the CPU/GPU boundary.

\end{abstract}

\section{Introduction, approach and outcomes (500 words)}

Increasing amounts of general computation is being offloaded to GPUs in soft
real-time applications such as games \cite{GPGPUTechniques2012}. Programs which
run on GPUs are called shaders, and running them has historically required
using high-level APIs like OpenGL and DirectX to manually pass shader
source-code to GPU drivers which then compile and run them at runtime. Although
modern graphics APIs like Vulkan \cite{Vulkan} operate in a similar manner,
SPIR-V, a new intermediate representation (IR) language is used to pass
pre-compiled shaders instead \cite{SPIRV}.

However, despite data structures being passed back-and-forth between the CPU and
the GPU, the type and error checking between the two platforms and their
respective languages has historically been non-existent. Furthermore, the APIs
require extensive boilerplate in order to set-up shader invocations, another
surface for errors.

Various projects and tools have developed which aim to manage these
complications at different levels. One approach is to develop domain specific
languages (DSLs) with high-levels of parallelism that allow them to be
automatically compiled for a wide variety of heterogeneous architectures
\cite{DSL1} \cite{DSL2} \cite{Theano2016}.

However, these systems are unsuitable for real-time applications which require
general purpose computation and fine-grained control of resource usage in order
to achieve deterministic and consistent performance - meaning that they have not
seen much uptake beyond scientific computing \cite{Theano2016}.

Another approach is take existing languages and extend them top-down with
support for heterogeneous architectures via annotated code, taken by C++ AMP
\cite{CAMP}, JCUDA \cite{JCUDA2009}, and more \cite{SYCL} \cite{HCC}
\cite{Theano2016} \cite{CParallel}, however so far these have suffered from the same inability
to offer the user fine-grained control that DSLs do, and the top-down approach
of making existing general-purpose languages semi-compatible with GPUs makes
them awkward to work with, meaning they have been unsuccessful in gaining
traction in industry \cite{CAMPFail1} \cite{CAMPFail2}.

The third approach that has been taken is to design a single general-purpose
language from the bottom-up with compile-time and runtime-systems built in to
allow for automatic resource management and workload distribution across
heterogeneous architectures \cite{Lime2012} \cite{Lime2010}. Although these
systems may represent the future of heterogeneous computing, they still have a
long way to go before seeing serious uptake.

My project will be to take a bottom-up approach, to design and implement a two
langauges, one for the CPU and one for the GPU - allowing for the context-aware
programming and fine-grained control that existing bottom-up or top-down
single-language systems do not provide. Furthermore, the bottom-up design will
allow for a unified type-system to allow for compile-time checking of common
errors which can frustrate current GPU programmers. Finally, syntactic sugar
for the explicit control of boilerplate generation at the language level can
help reduce another class of errors as well.

By the end I will have produced a language design and associated compiler that
can be used to compile the language, with suitable components being compiled
for a CPU and GPU backend.

\section{Workplan (500 words)}


This will be done iteratively,
by targetting individual components in the existing workflow and designing
language features which make them less cumbersome. The aims of this project are
to provide compile-time guarantees that certain errors on the inteface between
the GPU and CPU aren't present, so there shouldn't be any runtime cost.

My evaluation will consist of demonstrating which problems have been solved,
showing equivalent C/GLSL where they can be easily be introduced, and showing
how the new language  compile with them whilst still having the
similar runtime performance. As the focus of this isn't on compile-time
optimisations, the testing will be performed with compiler optimisations
disabled.
Some stuff here.

\bibliography{references}
\bibliographystyle{ieeetr}

\newpage
\appendix

\section{Time Table}

Counting from the 1st of December to the aimed submission date of the 27th of
November there are 25 weeks. So I will plan for 24 weeks of work, including a
grace week in case I fall behind.

\subsection{Weeks 1-2 (20 hours)}

Write 5 sample programs with a Vulkan backend in C/GLSL that can be used to
demonstrate the common issues a new language design could solve. It is
important that these programs use a subset of the features available in C/GLSL.
As the goal of the language is not to re-implement existing languages, but
demonstrate how a language could be designed to solve these common problems.
One of the programs can be for the CPU only for testing purposes.

\subsection{Weeks 3-4 (25 hours)}

'Port' the sample programs to what they would ideally look like in the
front-end of a new language. Send these off for feedback from the project
supervisor. Start designing a grammar for the language for parsing.

\subsection{Weeks 5-6 (30 hours)}

Implement a parser for the language and construct an AST of each example pro-
grams. Implement a type-checker that operates on the AST and demonstrate it can
catch the errors the can be easily demonstrated to occur. Have an intepreter be
able to run the AST for the CPU-only and give the same result as the
C-equivalent.

\subsection{Weeks 7-8 (20 hours)}

Compile the AST to bytecode.

\subsection{Weeks 9-10 (20 hours)}

Compile the CPU side bytecode to an LLVM backend.

\subsection{Weeks 11-12 (20 hours)}

Compile the GPU side bytecode to a SPIR-V backend.

\subsection{Weeks 13-14 (30 hours)}

Demonstrate that the sample programs compile and run.

\subsection{Weeks 15-16 (40 hours)}

Flesh-out and demonstrate more ideas. Use these two weeks to investigate
further directions this could go and implement any possible extensions. Polish
is the main focus here.

\subsection{Weeks 17-18 (40 hours)}

Perform the evaluation.

\subsection{Weeks 19-20 (40 hours)}

Write the first draft of the dissertation. Send it to my project supervisor for
feedback. Work on developing the software further in any spare time.

\subsection{Weeks 21-22 (40 hours)}

Write second draft of the dissertation.

Send it to my project supervisor for feedback. Work on developing the software
further in any spare time.

\subsection{Weeks 23-24 (40 hours)}

Write final draft of the dissertation. Send it to my project supervisor for
feedback. Work on developing the software further in any spare time. Submit
final project.

\section{Example of boilerplate elimination and error checking}

TODO:




\end{document}
