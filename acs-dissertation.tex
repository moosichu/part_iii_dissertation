%%
%% ACS project dissertation template.
%%
%% Currently designed for printing two-sided, but if you prefer to
%% print single-sided just remove ",twoside,openright" from the
%% \documentclass[] line below.
%%
%%
%%   SMH, May 2010.


\documentclass[a4paper,12pt,twoside,openright]{report}


%%
%% EDIT THE BELOW TO CUSTOMIZE
%%

\def\authorname{Tom M. Read Cutting\xspace}
\def\authorcollege{Downing College\xspace}
\def\authoremail{tr395@cam.ac.uk}
\def\dissertationtitle{Heterogeneous type checking in multi-language CPU-GPU systems}
\def\wordcount{TODO}


\usepackage{epsfig,graphicx,hyperref,parskip,setspace,tabularx,xspace}
\graphicspath{ {images/} }

%% START OF DOCUMENT
\begin{document}


%% FRONTMATTER (TITLE PAGE, DECLARATION, ABSTRACT, ETC)
\pagestyle{empty}
\singlespacing
\input{titlepage}
\onehalfspacing
\input{declaration}
\singlespacing
\newpage
{\Huge \bf Abstract}
\vspace{24pt}

% 1 sentence setting out the the scene, future holds promise.

% 2. Paper explores cross module checking between languages.

Programming for Graphical Processing Units (GPUs) can be frustrating and
error-prone due to the noncompatible nature of programming languages designed
for different architectures. GPU programmers have to to contend with manually
ensuring data structures and types are preserved on the boundaries between code
for the CPU and GPU - even when using high-level programming languages on each.
Although the future holds promise with unified languages which automatically
optimise for heterogeneous architectures, they are currently either domain
specific or lack the control necessary to achieve the desired performance
levels that can be achieved using traditional workflows.

This paper explores cross module type checking between languages for
heterogeneous computing systems. Although the ideas are generalisable, the
focus is on GPUs and CPUs. Two approaches are discussed, implemented and
evaluated. The first being a front-end preprocessor for C and GLSL using
traditional backends, the second being a completely operational pair of toy
languages with compatible type systems. Both of these approaches have their own
tradeoffs, with the goal of improving the user experience when interacting with
GPUs whilst still offering the high level of control traditional workflows
provide.

\newpage
\vspace*{\fill}


\pagenumbering{roman}
\setcounter{page}{0}
\pagestyle{plain}
\tableofcontents
\listoffigures
\listoftables

\onehalfspacing

%% START OF MAIN TEXT

\chapter{Introduction}
\pagenumbering{arabic}
\setcounter{page}{1}

% This is the introduction where you should introduce your work.  In
% general the thing to aim for here is to describe a little bit of the
% context for your work --- why did you do it (motivation), what was the
% hoped-for outcome (aims) --- as well as trying to give a brief
% overview of what you actually did.

% It's often useful to bring forward some ``highlights'' into
% this chapter (e.g.\ some particularly compelling results, or
% a particularly interesting finding).

% It's also traditional to give an outline of the rest of the
% document, although without care this can appear formulaic
% and tedious. Your call.

TODO

\chapter{Technical Background}

% A more extensive coverage of what's required to understand your
% work. In general you should assume the reader has a good undergraduate
% degree in computer science, but is not necessarily an expert in
% the particular area you've been working on. Hence this chapter
% may need to summarize some ``text book'' material.

% This is not something you'd normally require in an academic paper,
% and it may not be appropriate for your particular circumstances.
% Indeed, in some cases it's possible to cover all of the ``background''
% material either in the introduction or at appropriate places in
% the rest of the dissertation.

% This chapter covers relevant (and typically, recent) research
% which you build upon (or improve upon). There are two complementary
% goals for this chapter:
% \begin{enumerate}
%   \item to show that you know and understand the state of the art; and
%   \item to put your work in context
% \end{enumerate}

% Ideally you can tackle both together by providing a critique of
% related work, and describing what is insufficient (and how you do
% better!)

% The related work chapter should usually come either near the front or
% near the back of the dissertation. The advantage of the former is that
% you get to build the argument for why your work is important before
% presenting your solution(s) in later chapters; the advantage of the
% latter is that don't have to forward reference to your solution too
% much. The correct choice will depend on what you're writing up, and
% your own personal preference.

Graphics Processing Units (GPUs) were originally designed to be fixed-function
3D graphics hardware accelerators aimed at hobbyist gamers who wanted to play
the latest and greatest games with the prettiest graphics possible \cite{TODO}.
However, as video game developers wanted more artistic control over what was
rendered, the \textit{shader model} was introduced which gave graphics
programmers some control ove how computer graphics were rendered. Although
initially simple, as the desire for flexibility over these systems increased,
the capabilities of graphics cards themselves did so as well, reaching a point
where they are now essentially highly-parallel general-purpose computation
machines. A lot of the early (and current) development of GPUs has been driven
by a desire to increase the capabilities they offer to video games, allowing
them to drive not traditional rendering, but also physics \cite{TODO},
procedural world generation \cite{TODO}, post-processing effects \cite{TODO}
and alternative types of rendering including ray-tracing \cite{TODO}.

However, this increase in flexibility has allowed GPUs be used in increasingly
general domains, including scientific computing \cite{TODO}, crypto-currency
mining \cite{TODO}, video-processing \cite{TODO} and artificial intelligence
\cite{TODO}. To the point where the hobbyists they cards were originally
designed for are being priced out of the market \cite{TODO}.

However, despite the exploding number of use cases.


\chapter{Design}

TODO:


\chapter{Implementation}

% This chapter may be called something else\ldots but in general
% the idea is that you have one (or a few) ``meat'' chapters which
% describe the work you did in technical detail.

TODO:


\chapter{Evaluation}

% For any practical projects, you should almost certainly have
% some kind of evaluation, and it's often useful to separate
% this out into its own chapter.

TODO:

\chapter{Conclusion and Further Work}

% As you might imagine: summarizes the dissertation, and draws
% any conclusions. Depending on the length of your work, and
% how well you write, you may not need a summary here.

% You will generally want to draw some conclusions, and point
% to potential future work. \cite{DirectXWorkings}

TODO:

\appendix
\singlespacing

\bibliographystyle{unsrt}
\bibliography{references}

\end{document}
