\newpage
{\Huge \bf Abstract}
\vspace{24pt}

% 1 sentence setting out the the scene, future holds promise.

% this thesis sets scene then give contributions.

% TODO: so step back and look at heterogenous computing

% TODO: because compiled in seperate languages, use better words

% 2. Paper explores cross module checking between languages.

Programming for heterogeneous systems such as GPUs can be frustrating and
error-prone due to the incompatible nature of programming languages designed
for different architectures. Programmers have to contend with manually ensuring
data structures and types are consistent on the boundaries between code for
different systems - even when targeting them with high-level and
internally-sound programming languages.

There are multiple approaches that can be used to ease this burden. For
example, the future holds promise with unified languages (TODO such as) which
automatically target heterogeneous architectures, they are currently either
domain specific or lack the control necessary to achieve the desired
performance levels that can be achieved using traditional workflows.

TODO: tighten up below. talk about the the two solutions very briefly

This paper explores mechanisms for cross-module type checking between languages
for heterogeneous computing systems. Although the ideas are generalisable, the
focus is on GPUs and CPUs. Two approaches are discussed, implemented and
evaluated. The first being a front-end preprocessor for C and GLSL using
traditional backends, the second being a completely operational pair of toy
languages with compatible type systems. Both of these approaches have their own
tradeoffs, with the goal of improving the user experience when interacting with
GPUs whilst still offering the high level of control that traditional workflows
provide.

\newpage
\vspace*{\fill}
